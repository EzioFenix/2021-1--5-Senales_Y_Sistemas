\subsubsection{Sistema 2}

\textbf{2. ¿Cómo son las raices del polinomio característicos del sistema en estudio? justifique su respuesta.}

Podría ser un sistema de primer o segundo orden críticamente amortiguado, en este caso tomamos que es de segundo orden y sabemos que   $\zeta= 1$ y por tanto sus raices serían reales, repetidas y negativas.\\


\textbf{3. De acuerdo a la gráfica mostrada en la Fig. 43, defina una ecuaci´on matemática que caracterice el comportamiento de dicho sistema.}

La ecuación general para un sistema de segundo orden es

$$
\frac{\mathrm{d}^{2} y(t)}{\mathrm{d} t^{2}}+2 \zeta \omega_{n} \frac{\mathrm{d} y(t)}{\mathrm{d} t}+\omega_{n}^{2} y(t)=k \omega_{n}^{2} x(t)
$$

Pero la reescribimos, ya que la entrada es una señal es $x(t)$ y es un escalón es u(t) para $\zeta=1$, quedando como resultado:

$$
\frac{\mathrm{d}^{2} y(t)}{\mathrm{d} t^{2}}+2 \omega_{n} \frac{\mathrm{d} y(t)}{\mathrm{d} t}+\omega_{n}^{2} y(t)=k \omega_{n}^{2} u(t)
$$

\section{OBSERVACIONES Y CONCLUSIONES}


\textbf{Alfaro Domínguez Rodrigo:} Esta práctica nos ayudó a complementar los conocimientos adquiridos en la anterior, ya que nos centramos en el análisis de la respuesta escalón de distintos sistemas físicos. Gracias a esto pudimos entender mejor como diseñar un sistema físico para obtener algún comportamiento deseado. De la misma forma pudimos observar el comportamiento de un sistema representado por una ecuación diferencial de segundo orden.\\

\textbf{ Barrera Peña Víctor Miguel:} En la práctica se planteó para que su pudiera corroborar que los alumnos pudiéramos identificar los sistemas al sólo verlos y deducir las cualidades del sistemas, y relacionar ello para entender cierto tipo de sistemas y su comportamiento, pensemos que es como cuando nos introdujimos a cálculo en el que podíamos predecir un comportamiento matemático y decir cual era la tendencia, aquí es el mismo caso, pero ahora a parir de un sistema físico, al haber entendido dicho propósito e identificar a las gráficas que se presentaron y deducir sus características, yo puedo afirmar que se cumplió exitosamente la práctica.\\


\textbf{Villeda Hernández Erick Ricardo:} En esta práctica pudimos observar diferentes comportamientos característicos de algunos sistemas físicos, con los cuales trabajamos. Esto a partir de su respuesta al escalón. Uno de los sistemas con el cual trabajamos teóricamente fue uno en donde observamos la respuesta del sistema de segundo orden al escalón y otro sistema en la actividad 3 fue un spinner en donde observamos en cada ejercicio que pueden ocurrir 4 casos distintos según sus componentes de cada sistema.
