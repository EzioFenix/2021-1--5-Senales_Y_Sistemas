\noindent \justifying

\section{Previo}

\subsection{Identificar un sistema dinámico que se tenga en casa y definir la salida y la entrada del mismo (para discusión en clase)}
\subsection{¿Como analizaría un sistema de orden mayor?}
Para analizar un sistema de orden superior empezamos por escribir su función de transferencia:
\begin{equation}
H(s)=K\frac{(s-z_1)(s-z_2)...(s-z_n)}{(s-p_1)(s-p_2)(s-p_n)}
\end{equation} 
La mayor parte de la información de como funciona el ssitema nos la darán la localización de los polos y ceros. Esto determina si el sistema es estable o no.\\
En caso de tener polos reales la ecuación toma la siguiente forma:
\begin{equation}
H(s)=\frac{a_1}{s-p_1}+...+\frac{a_n}{s-p_n}
\end{equation}
A partir de aquí analizamos su respuesta a un impulso y un escalón, quedándonos sus ecuaciones de una de las siguientes formas respectivamente:
\begin{equation}
y_{imp}=\alpha_1 e^{p_1t}+..+\alpha_n e^{p_nt}
\end{equation}
\begin{equation}
y_{step}=\beta_0 + \beta_1 e^{p_1t}+...+ \beta_n e^{p_nt}
\end{equation}
Cada polo real p genera un término exponencial en la respuesta. El comportamiento de las osilaciones va a depender de si la parte real del poplo es negativa o positiva, mientras que la magnitud depende de los ceros.\\
En el caso de un sistema de segundo orden podemos escribir su ecuación característica en términos de zeta y omega, de la siguiente forma:
\begin{equation}
\frac{d^2y(t)}{dt^2}+2\zeta \omega_n \frac{dy(t)}{dt}+(\omega_n)^2y(t)=k(\omega_n)^2x(t)
\end{equation}
A partir de su respuesta en la ecuación homogénea podemos llegar a un polinimio de la siguiente forma:
\begin{equation}
s^2+2\zeta\omega_ns+\omega^2_n=0
\end{equation}
La respuesta del sistema va a depender de los valores que tenga el término $\zeta$, siendo sus valores posibles entre cero e infinito positivo. Lo que nos interesará para el diseño de un sistema es que su valor sea mayor o igual a uno.
\subsection{¿Cuál es la importancia de la constante de tiempo $\tau$ y el factor de amortiguamiento $\zeta$ ?}

