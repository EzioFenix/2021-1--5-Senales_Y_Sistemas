\documentclass{article}
\usepackage[spanish]{babel}


\begin{document}	
\noindent\textbf{2. ¿Qué sistemas con elementos disipadores conoce?}

El disipador es el elemento que se encarga de disipar o de eliminar el calor que genera un componente electrónico debido al uso. Se utiliza en sistemas \textbf{eléctricos, mecánicos traslacionales y mecánicos rotacionales}. Todos los disipadores se pueden clasificar en dos categorías: activos y pasivos.\\
\textbf{Disipadores Activos:}
Estos suelen tener un ventilador de alguna clase, siendo los de rodamientos y motor los más comunes. Su rendimiento es excelente, pero son más bien caros al tener partes móviles.\\
\textbf{Disipadores Pasivos:}
Estos no tienen componentes mecánicos, y emplean solo la convección para disipar la energía térmica. Al no tener partes móviles, son más fiables, pero es necesario que el aire circule por las aletas.\\
\textbf{Algunos sistemas con elementos disipadores son:}\\
	-Se emplea sobre transistores en circuitos electrónicos de potencia para evitar que las altas temperaturas puedan llegar a 		    quemarlos.\\
	-En las computadoras su uso es intensivo y prolongado,ya que sirve para que algunas tarjetas gráficas o el microprocesador 			puedan disminuir sus altas temperaturas.\\
	-Otro sistema en donde se utilizan los disipadores son en las consolas de videojuegos.
\end{document}