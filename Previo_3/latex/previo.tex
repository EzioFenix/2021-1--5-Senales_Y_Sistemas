{\Huge Previo }

\noindent \justifying


\section{¿Qué métodos se pueden utilizar para resolver ecuaciones en diferencias en el dominio del tiempo discreto?} 
\section{¿Cuál es la relación entre las variables s y z? ¿Cómo se relaciona el plano complejo en s con el plano complejo en z?}
\section{¿Cómo se caracteriza la estabilidad de los sistemas lineales e invariantes de tiempo discreto?}
\section{¿Qué diferencias existen entre los métodos de fracciones parciales para sistemas de tiempo continuo y sistemas de tiempo discreto?}
Ambos métodos son usados para obtener la anti transformada de Laplace y Z respectivamente. En el caso de sistemas de tiempo discreto el método es mejor conocido como expansión de fracciones parciales, y difiere con el método de sistemas continuos en los siguientes aspectos:
\section{En qué dispositivo de la vida cotidiana se realizan conversiones de señales de tiempo continuo a tiempo discreto y viceversa?}

\section{Preguntas de cierre}

\subsection{Explique brevemente la importancia de la conversión de señales de tiempo continuo a tiempo discreto}
\subsection{¿Qué relación existe entre la transformadas de Laplace y Z?}
\subsection{¿Cómo se caracteriza la estabilidad de sistemas de tiempo continuo y tiempo discreto en el contexto de funciones de transferencia?}
