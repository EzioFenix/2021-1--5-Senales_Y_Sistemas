\documentclass{article}
\usepackage[spanish]{babel}


\begin{document}	
	\noindent\textbf{4. ¿Qué diferencias existen entre los métodos de fracciones parciales para sistemas de tiempo continuo y sistemas de tiempo discreto?}
	\newline
Ambos métodos son usados para obtener la anti transformada de Laplace y Z respectivamente. En el caso de sistemas de tiempo discreto el método es mejor conocido como expansión de fracciones parciales, y difiere con el método de sistemas continuos en los siguientes aspectos:\\
	    1. Cada fracción va multiplicada por un factor z que facilita su anti transformación.\\
	    2. Se considera un primer coeficiente que no va acompañado de ningún factor calculado de las siguiente forma:\\
	    \[
        d_{0}=\frac{b_{m}} {(−p_{1})(−p_{2})...(−p_{n})}
        \]
        Donde m=número de ceros y n=numerode polos.\\
        
El método de fracciones parciales para sistemas de tiempo discreto es idéntico al que se utiliza en la transformada de Laplace (tiempo continuo), y requiere que todos los términos de la expansión en fracciones parciales se puedan reconocer fácilmente en la tabla de pares de transformadas Z.\\
Si X(z) tiene uno o más ceros en el origen (z = 0), entonces X(z)/z ó X(z) se expande en la suma de términos sencillos de primer o segundo orden mediante la expansión en fracciones parciales, y se emplea una tabla de transformadas Z para encontrar la función del tiempo correspondiente para cada uno de los términos expandidos.\\
	    
Del mismo modo que en sistemas de tiempo continuo y teniendo en cuenta la fracción:\\
\[
F(z)=\frac{N(z)}{D(z)}=\frac{N(z)}{(Z+P_{1})(Z+P_{2})(Z+P_{3})...(Z+P_{i})}
\]
Donde P1, P2, P3 ... Pi son las raíces del polinomio, estas raíces podrán ser: reales simples, reales múltiples, complejas simples, complejas múltiples.
	
\end{document}