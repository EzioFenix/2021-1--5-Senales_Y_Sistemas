\documentclass{article}
\usepackage[spanish]{babel}
\usepackage{mathtools}


\begin{document}	
	\noindent\textbf{3. ¿Cómo se caracteriza la estabilidad de los sistemas lineales e invariantes de tiempo discreto?}
	
	Un sistema discreto es BIBO si para cada entrada de $x[n]$ está acotada, de forma que si existe una $M<\infty$ que cumpla con que $|x[n]|>M$ para tona n, y cada salide $y[n]$ está igualmente acotada.
	
	\begin{equation}
	|x[n]| \leq M < \infty
	\end{equation}
	
	La anterior ecuación nos indica que va a existir un valor M que va a acotar a todos los valores x[n] de la señal de entrada.
	
	En cambio, un sistema de tiempo discreto LTI es BIBO estable si y solo si su secuencia de respuesta de impulso {h [n]} es absolutamente sumable, es decir,
	\begin{equation}
		S=\sum_{n=-\infty}^{\infty}|h[n]|<\infty
	\end{equation}
	
	\noindent\textbf{Demostración:}

	Partimos de lo siguiente
	\begin{equation}
		|x[n]| \leq M_x < \infty
	\end{equation}		
	
	Empecemos considrando a la salida de la señal como una función convolución
	\begin{equation}
		y[n]=x[n]*h[n]=\sum_{m=-\infty}^{\infty}x[m]h[n-m]
	\end{equation}
	
	Si aplicamos valor absoluto a ambos lados de la y usamos la deisgualdad de ters componentes en la suma obtenemos: 
	\begin{equation}
		\lvert y[n]\rvert = \lvert \sum_{m=-\infty}^{\infty}x[m]h[n-m] \rvert \leq \sum_{m=-\infty}^{\infty}x[m]h[n-m]
	\end{equation}		
	\begin{equation}
		=\sum_{m=-\infty}^{\infty}\lvert x[m]h[n-m] \rvert \leq \sum_{m=-\infty}^{\infty}M_x \lvert h[n-m] \rvert
	\end{equation}
	\begin{equation}
		\leq M_x \sum_{m=-\infty}^{\infty} \lvert h[n-m] \rvert = M_xS
	\end{equation}

	Si $S<\infty$ tenemos que 
	
	\begin{equation}
		|y[n]|\leq B_y < \infty
	\end{equation}
	
	Para demostrar la convergencia de S	consideramos la entrada x[n] de la siguiente forma
	
	\[
	x[n]=
	\begin{cases}
		$sgn(h[-n])$, & \text{si h[-n]  $\neq$ 0} \\
		K, & \text{si h[-n] = 0}				\end{cases}
	\]
	
	Donde sgn(c)=1 si c>0 y sgn(c)=-1 si c<0 y |K| $\leq$ 0
	\newline
	
	Si usamos la entrada n=0 tenemos:
	\begin{equation}
		y[0]=\sum_{k=-\infty}^\infty sng(h[k]) h[k] = S \leq B_y < \infty
	\end{equation}
	Con lo que se demuestra que $|y[n]| \leq B_Y$ que implica que $S\leq\infty$

	\textbf{Bibliografia:}	
	https://web.njit.edu/~akansu/Ch2(3)Handouts_3e.pdf
	https://cnx.org/contents/KilsjSQd@10.18:9kZ-CT3d@1/Causality-and-Stability-of-Discrete-Time-Linear-Time-Invariant-Systems
	
\end{document}
