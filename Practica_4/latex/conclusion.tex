\subsection{Solución actividad 5}
1. Identificar el numero de elementos que almacenan energía:\\
El sistema presenta cuatro almacenadores de energía. La ecuación resultante va a ser de orden dos, debido a que el sistema presenta los dos tipos de elementos almacenadores (de flujo y de esfuerzo), siendo estos elementos los siguientes:\\
\begin{itemize}
\item Almacenadores de esfuerzo: Inductores L y Lm
\item Almacenadores de flujo: Capacitor C y la masa rotada.
\end{itemize}
2. Identificar el número de restricciones físicas tanto de compatibilidad como de continuidad.\\
El sistema presenta cuatro restricciones físicas y se dividen de la siguiente forma: los dos circuitos en serie presentan restricciones de compatibilidad, mientras que la conexión en paralelo y la masa presentan restricciones de tipo de compatibilidad.\\
3. Plantear las restricciones físicas encontradas:\\
De la primera malla obtenemos el siguiente análisis:\\
\begin{equation}
E=V_L+v
\end{equation}
Si analizamos uno de los nodos que conecta a la primera malla con la conexión en paralelo obtenemos:\\
\begin{equation}
i=i_c+i_r+i_a
\end{equation}
De la segunda malla obtenemos el siguiente análisis:
\begin{equation}
v=V_{RM}+V_{LM}+VM
\end{equation}
Y por último, del sistema de la masa obtenemos lo siguiente:
\begin{equation}
K_{ew}=J\frac{d}{dt}w+Bw-J_L
\end{equation}
4. Sustituir las relaciones constitutivas de los elementos
Sustituyendo las relaciones constitutivas de los elementos de la primera malla obtenemos:
\begin{equation}
E=L\frac{d}{dt}i(t)+w
\end{equation}
En cuanto a los elementos constitutivos del análisis de la entrada de corriente al nodo alfa queda:
\begin{equation}
i=C\frac{dv}{dt}+\frac{v}{R}+i_a
\end{equation}
Y finalmente podemos expresar el análisis de la segunda malla de la siguiente forma:
\begin{equation}
v=R_mi_a+L_m\frac{d}{dt}i_a+K_{ew}
\end{equation}
5. Obtener el modelo matemático de forma matricial.
La representación del modelo matemático en forma matricial responde a la siguiente forma:
\begin{equation}
x'(t)=A(y)x(t)+B(t)u(t)
\end{equation}
La cual la podemos denotar como:
\begin{equation}
x'=Ax+Bu
\end{equation}
Donde $x'$ es el vector de variables de estados, $A$ es la matriz de estados, $x$ es el vector de estados, $B$ es la matriz de entrada y $u$ es el vector de entrada.\\
Para poder usar la expresión anterior debemos de dejar las ecuaciones de las restricciones de nuestro sistema en función de sus derivadas normalizadas. Po lo cual las ecuaciones a usar quedarán de las siguientes:
\begin{equation}
i'=-\frac{1}{L}v+\frac{1}{L}E
\end{equation}
\begin{equation}
v'=\frac{1}{C}i-\frac{1}{RC}v-\frac{1}{C}i_a
\end{equation}
\begin{equation}
i'_a=\frac{1}{L_m}v-\frac{1}{L_m}R_mi_a-\frac{K_e}{L_m}w
\end{equation}
\begin{equation}
w'=\frac{1}{J}k_ei_a-\frac{1}{J}B_w
\end{equation}
Ya teniendo las ecuaciones despejadas podemos sustituir en la ecuación matricial obteniendo la siguiente expresión:\\
llflfld
\begin{equation}
\begin{pmatrix}
i'\\
v'\\
i_a'\\
w' 
\end{pmatrix}
=
\begin{pmatrix}
0 & -\frac{1}{L} & 0 & 0\\
\frac{1}{C} & -\frac{1}{RC} & -\frac{1}{C} & 0\\
0 & \frac{1}{L_m} & -\frac{R_m}{L_m} & \frac{K_e}{L_m}\\
0 & 0 & \frac{K_e}{J} & -\frac{B}{J}
\end{pmatrix}
\begin{pmatrix}
i\\
v\\
i_a\\
w
\end{pmatrix}
+
\begin{pmatrix}
\frac{1}{L}\\
0\\
0\\
0\\
\end{pmatrix}
E
\end{equation}
6. ¿Qué se puede concluir del sistema físico obtenido?//
Que la unión de distintos sistemas físicos nos da como resultado la unión de las ecuaciones de dichos sistemas, es decir, su modelo matemático va a ser la suma de los modelos matemáticos de cada uno de los subsistemas que conforman a dicho sistema.

\section{OBSERVACIONES Y CONCLUSIONES}

	Villeda Hernandez Erick Ricardo: En la realización de esta práctica aplicamos los conocimientos teóricos de sobre el modelado de sistemas físicos (eléctricos, mecánicos translacional y mecánicos rotacionales), los cuales nos ayudaron a conocer y a diferenciar los elementos principales de todo sistema físico como los almacenadores de flujo, esfuerzo, los disipadores y la fuente de energía.Ya que dependiendo del tipo de sistemas con el que estemos trabajando va a variar su elementos. Se trabajó parte del funcionamiento de un sistema híbrido y se analizaron  las restricciones de constitución, continuidad, y compatibilidad para poder obtener un modelo matemático más apegado a la realidad.
