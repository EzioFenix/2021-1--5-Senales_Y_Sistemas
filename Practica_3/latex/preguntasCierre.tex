\section{Preguntas de cierre}

\subsection{Explique brevemente la importancia de la conversión de señales de tiempo continuo a tiempo discreto}

Al transformar una señal de tiempo continuo a tiempo discreto (muestrear una señal) lo que nos permite es tomar valores específicos para poder analizar dicha señal de entrada. Por otra parte, la discretización también nos sirve para transformar una señal de análoga a digital. La conversión de la señal análoga en señal digital se realiza, entre otras razones porque las señales digitales presentan grandes ventajas a la hora de ser transmitidas y/o procesadas, como por ejemplo mayor inmunidad al ruido o mayor facilidad de procesamiento.\\
En las aplicaciones tecnológicas el muestreo se toma en intervalos de tiempo iguales, proceso denominado “Muestreo periódico de la señal”, lo que facilita procesos como el de la reconstrucción de una señal.


\subsection{¿Qué relación existe entre la transformadas de Laplace y Z?}
Ambas transformadas son una herramienta de gran alcance formulada para solucionar una variedad amplia de problemas de valor inicial. La estrategia es transformar las ecuaciones diferenciales difíciles en los problemas simples del álgebra donde las soluciones pueden ser obtenidas fácilmente.\\
La diferencia que existe entre la transformada de Laplace y la transformada Z es que la transformada de laplace se emplea en el estudio de los sistemas continuos lineales e invariantes en el tiempo y la transformada z se utiliza en el análisis de los sistemas discretos lineales e invariantes en el tiempo.
		
\subsection{¿Cómo se caracteriza la estabilidad de sistemas de tiempo continuo y tiempo discreto en el contexto de funciones de transferencia?}

En sistemas continuos la estabilidad se analiza en el denominador de la función de transferencia. La estabilidad va a depender de la posición de las raíces obtenidas al utiliar el método de fracciones parciales. En el caso de raices positivas observamos lo siguiente:

\begin{equation}
H(s)=\frac{C_1}{s+p_1}+\frac{C_2}{s+p_2}+...+\frac{C_N}{s+p_N}
\end{equation}

A dicha transformada de Laplace le corresponde la siguiente antitransformada:

\begin{equation}
h(t)=C_1e^{-p_1t}+C_2e^{-p_2t}+...+C_Ne^{-p_Nt}
\end{equation}

Cada termino de la siguiente ecuación puede caer en uno de los siguientes tres casos:

\begin{itemize}
\item La raíz $s=-p_n$ es positiva, entonces el exponente es positivo y la exponencial tenderá al infinito con el tiempo.
\item La raíz $s=-p_n$ es negativa, por lo que el exponente correspondiente es negativo y al exponencial tenderá a cero con el tiempo.
\item La raíz $s=-p_n$ es cero con lo cual la exponencial evaluada será uno obteniendo un valor constante $C_n$
\end{itemize}

De la misma forma se puede analizar los casos en que las raices de la transformada de Laplace sean complejas. En dichos casos la función de transferencia tendra una forma similar a la siguiente:

\begin{equation}
H(s)=\frac{C_1}{s+p_1}+\frac{C_2^2+C_3}{s^2+\alpha s+\beta}+...+\frac{C_N}{s+p_N}
\end{equation}

En este caso el termino complejo es la segunda fracción de la ecuación. Si analizamos la antitransformada de laplace de dicho término podemos ver que puede tener una de las siguientes dos formas:

\begin{equation}
\frac{\omega}{(s+a)^2+{\omega}^2} => e^{-\alpha t}seno(\omega t)
\end{equation}

\begin{equation}
\frac{s+a}{(s+a)^2+{\omega}^2} => e^{-\alpha t}cos(\omega t)
\end{equation}

Debido a que las funciones seno y coseno oscilan entre los valores de 1 y -1 el comportamiento del término imaginario va a depender de $e^-{\alpha t}$. Con esto podemos llegar a una de las siguientes conclusiones:

\begin{itemize}
\item Si la parte real del par conjugado es positiva las exponenciales asociadas tenderán en el tiempo a infinito.
\item Si la parte real del par conjugado es negativa las exponenciales asociadas tenderán en el tiempo a cero
\item Si la parte real del par conjugado es cero las exponenciales adquieren un valor constante.
\end{itemize}

Por lo tanto, en ambos casos, podemos decir que la estabilidad de un sistema va a depender de la posición de la parte real de los polos contenidos en la función de transferencia. Si la parte real es negativa es estable, si es negativa es inestable y si es cero la respuesta será críticamente estable por lo que se necesitará de otro análisis para determinar su estabilidad.

En cuanto a un señal discreta esta será acotada si y sólo si cualquier entrada acotada produce una salida acotada, es decir si

\begin{equation}
|x[n]|<\infty \forall n
\end{equation}

entonces 

\begin{equation}
|y[n]|<\infty \forall n
\end{equation}

Si se tiene un sistema lineal invariente en el tiempo su respuesta de estado cero está dada por:

\begin{equation}
y[n]=\sum_{m=-\infty}^n x[m]h[n-m]
\end{equation}

Si realizamos el cambio de variable k=n-m cuando $m=-\infty$ y cuando m=n en k=0, entonces

\begin{equation}
y[n]=\sum_{m=-\infty}^n h[k]x[n-m]
\end{equation}

En caso de que la señal de entrada sea acotoda tenemos lo siguiente

\begin{equation}
y[n]=\sum_{m=-\infty}^n h[k]x[n-m] \lq \sum_{m=-\infty}^n |h[k]x[n-m]| \lq M\sum_{m=-\infty}^n |h[k]|
\end{equation}

En caso de que la suma de la respuesta al impulso converja inferimos que 

\begin{equation}
|y[n]|<\infty
\end{equation}

que representa la condición necesaria para la estabilidad