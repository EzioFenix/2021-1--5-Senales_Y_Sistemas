Considere un circuito RLC como el mostrado en la Figura 23, cuyo comportamiento, considerando como entrada el voltaje $V_g(t)$ de la fuente y como salida el voltaje en el capacitor $V_c(t)$, está dado por la ecuación diferencial de segundo orden

\begin{equation}
	\frac{d^2V_c(t)}{dt^2}+\frac{R}[L}+\frac{1}{LC}V_g(t)=\frac{1}{LC}V_g(t)
\end{equation}

\textbf{Solución analítica del circuito:}

\textbf{Representación grafica: }


\begin{itemize}
	\item Considerando un periodo de muestreo de $ T_s = 1 $ y utilizando el método de discretización mediante
	diferencias finitas, encuentre la ecuación en diferencias asociada y resu´elvala utilizando el método de recurrencia. Compare los resultados gr´aficos de la versi´on de tiempo continuo y la de tiempo discreto para diferentes valores del periodo de muestreo (disminúyalo en un punto decimal hasta $T_s = 0,0001$).
\end{itemize}

\textbf{Comparación entre solución de tiempo discreto y aproximación con diferencias finitas para diferentes valores del tiempo de muestreo}

\begin{itemize}
	\item Obtenga la función de transferencia del sistema de tiempo continuo.
	\item Utilizando $ T_s = 1 $:
\end{itemize}

\textbf{A)} Obtenga la función de transferencia de tiempo discreto de la ecuación en diferencias que resultó en	el punto anterior.

\textbf{B)}  Obtenga la función de transferencia de tiempo discreto a partir de la función de transferencia de tiempo continuo del sistema utilizando un diferenciador discreto, ¿cómo son las funciones de transferencia obtenidas en este punto y el anterior? ¿qué puede concluir?


\begin{itemize}
	\item Grafique en una sola figura la respuesta al impulso del sistema de tiempo continuo, y las dos aproximaciones
	de tiempo discreto.
\end{itemize}

\textbf{Comparación de solución de tiempo contunuo y solución de tiempo discreto utilizando funciones de tranferencia}

\begin{itemize}
	\item Disminuya el tiempo de muestreo hasta obtener una aproximación adecuada de la respuesta del sistema
	y grafique la comparación. ¿Qu´e aproximación resultó mejor?
\end{itemize}

\textbf{Comparacion de solución de tiempo continuo y solución de tiempo discreto utilizando funciones de tranferencia para diferentes valores de tiempo de muestreo }

{\Large Control discreto de un sistema de tiempo continuo.}

Considere un sistema lineal e invariante en el tiempo representado por la siguiente funci´on de transferencia

\begin{equation*}
	G(s)=\frac{1}{s(s+3)}
\end{equation*}

\begin{itemize}
	\item Determine la estabilidad del sistema.
	\item Utilizando el software especializado de su preferencia, determine la respuesta al escalón del sistema y describa como es su comportamiento.
\end{itemize}

\begin{figure}[H]
	\centering
	\includegraphics[scale=0.7]{img2/fig24}
	\label{fig:fig24}
\end{figure}

\textbf{Respuesta al escalón del sistema a controlar}

\begin{itemize}
	\item Cuando se desea cambiar el comportamiento de un sistema se debe implementar un controlador de lazo
	cerrado, el cual compara la se˜nal de salida del sistema con la se˜nal de referencia y con base en esta se˜nal
	de error calcula la entrada del sistema para que se obtenga el comportamiento deseado, de acuerdo con
	el diagrama de bloques mostrado en Figura 24. El modo m´as simple de control consiste en el control
	proporcional, el cual realimenta un término proporcional del error de salida, es decir,
\end{itemize}

\begin{equation*}
	u_c=K(r-y)
\end{equation*}

La conexión de la Figura 24 se denomina conexión en retroalimentación negativa, y es posible determinar la función de transferencia correspondiente mediante software especializado, para lo cual se deben definir
previamente las funciones de transferencia del controlador, del sistema y del sensor. Considerando la función de transferencia del sistema, la del controlador como C(s) = K y la del sensor H(s) = 1, determine
la función de transferencia de lazo cerrado Gc(s) correspondiente. ¿Cómo son los polos del sistema? ¿Qué puede decir de la estabilidad del mismo?

\begin{itemize}
	\item A partir de las funciones de transferencia de lazo abierto y de lazo cerrado en tiempo continuo obtenga las versiones de tiempo discreto. Realice lo anterior utilizando los procedimientos presentados en la
	Introducción Teórica y el software especializado de su elección. Reporte sus resultados a continuación.
\end{itemize}

\textbf{Respuesta al escalón del sistema con control}

\begin{itemize}
	\item Determine los polos de lazo abierto y de lazo cerrado de tiempo discreto y caracterice la estabilidad de
	cada uno de estos. Determine la respuesta al escalón de ambos sistemas utilizando software especializado.
	Escriba sus resultados a continuación y las gráficas obtenidas en los espacios correspondientes.
\end{itemize}

\textbf{Respuesta al escalón del sistema en tiempo discreto}

\textbf{Respuesta al escalón del sistema de control en tiempo discreto}