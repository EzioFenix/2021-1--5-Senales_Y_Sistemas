\section{Observaciones y conclusiones}

\begin{itemize}
	\item \textbf{Alfaro Domínguez Rodrigo:}
	Al final de esta práctica logramos entender las características que determinan la estabilidad de un sistema al igual que como pasar una señal de tiempo continuo a tiempo discreto, esto con la finalidad de facilitar su analisis con una señal similar a la señal real, con ayuda de Matlab. De la misma forma pudimos comprender el comportamiento y la utilidad de un sistema de lazo cerrado.
	\item \textbf{Barrera Peña Víctor Miguel:} 
	Si bien, se logró realizar la práctica e identificar las señales básicas que marca la práctica, además de muestrear dichas señales , no creo que en todos los aspectos haya tenido una clara resolución de cada una de las actividades, siento que se terminó la práctica con prisa y dejando algunas dudas de qué era lo que tenía que realizar el inciso, pero general la resolución fue casi total, dejando un margen pequeño de dudas, aunque sigo teniendo dudas de algunos comandos en Matlab, pero ellos por la misma naturaleza del software que nunca lo he utilizado antes. otro punto a destacar es que siento que hay un desfase mayor en esta materia con respecto a teoría, por ello la transformada Z sigue siendo un poco indiferente ante la resolución en esa práctica, sin embargo puedo decir que esta práctica se concluyó con éxito i satisface los estándares que se marcan.
	\item \textbf{Villeda Hernández Erick Ricardo:}
	En esta práctica logramos trabajar con algunas señales, las cuales con ayuda de MATLAB nos sirvieron para conocer la relación que existe entre las señales físicas  y su representación matemática. Así como la forma y la importancia de discretizar un sistema. Por otra parte logramos determinar la función de transferencia de un sistema y la función de transferencia de lazo cerrado, la cual nos sirvió para verificar la inestabilidad de un sistema dependiendo el valor K de entrada. Por último logramos conocer la importancia de un controlador en un sistema, ya que este controla la salida del mismo.

	
\end{itemize}