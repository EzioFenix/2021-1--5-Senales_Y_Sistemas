\section{Observaciones y conclusiones}

\begin{itemize}
	\item \textbf{Alfaro Domínguez Rodrigo:}
	Al final de esta práctica logramos entender las características que determinan la estabilidad de un sistema al igual que como pasar una señal de tiempo continuo a tiempo discreto, esto con la finalidad de facilitar su analisis con una señal similar a la señal real, con ayuda de Matlab. De la misma forma pudimos comprender el comportamiento y la utilidad de un sistema de lazo cerrado.
	\item \textbf{Barrera Peña Víctor Miguel:} 
	\item \textbf{Villeda Hernández Erick Ricardo:}
	En esta práctica logramos trabajar con algunas señales, las cuales con ayuda de MATLAB nos sirvieron para conocer la relación que existe entre las señales físicas  y su representación matemática. Así como la forma y la importancia de discretizar un sistema. Por otra parte logramos determinar la función de transferencia de un sistema y la función de transferencia de lazo cerrado, la cual nos sirvió para verificar la inestabilidad de un sistema dependiendo el valor K de entrada. Por último logramos conocer la importancia de un controlador en un sistema, ya que este controla la salida del mismo.

	
\end{itemize}